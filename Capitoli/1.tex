\chapter{Part 1} % (fold)
\label{cha:part_1}


As first step starting from the gain requirements we choose the lenght of transistors $M_1,M_2,M_3,M_4,M_6,M_7.$

Splitting the gain in half beetween the 2 stages we get\footnote{The evaluation of the second stage's gain has been done exactly in the same way}:

\begin{equation}
	G_{1^{st} stage} = 100 = \frac{2V_AL_1L_3}{V_{ov_1}L_{min}(L_1+L_3)}
\end{equation}


In this first step we decided to keep $L_1=L_2=L_3=L_4$ and $L_6=L_7$.

Respecting the constrait of $V_{ov_x} > 200mv$ we choose:

\begin{itemize}
	\item $V_{ov_1}=V_{ov_2}= 0.2V$ in order to keep the $g_m$ as high as possible in order to respect the bandwith specification\footnote{This will be also good for phase margin}.
	\item $V_{ov_3}=V_{ov_4}= 0.3V$ in this way we will have $g_{m3}=g_{m4}<g_{m1}=g_{m2}=$, squeezing the input referred noise.
	\item $V_{ov_5}= 0.4V$ to minimize the area\footnote{Accepting some losses in terms of CMRR}.
	\item $V_{ov_6}=V_{ov_7}= 0.2V$ in order to keep o good output voltage swing. 
\end{itemize}

Then starting from the bandwith and the phase margin specifications we tune the transconductance of the input pair and the value of the compensation capacitance, going through this calculation:

\begin{equation}
	GBWP=40MHz=\frac{g_{m1}}{2 \pi C_c}
\end{equation}
Assuming for this step $C_c = 1pF$
Then we check if the choosen values keep us in the stability contraint.

\begin{equation}
	C_c> \frac{g_{m1}}{g_{m6}}(C_1+C_L)
\end{equation}

During all the design we are going to assume a load capacitance of 1pF.
At this moment the M6's were not yet fixed , but for this calcultaion we assume $C_1+C_L=2pF$\footnote{We will check at the end that this assumption results confirmed.}

\begin{equation}
	\phi_m=180- actg(\frac {GBWP} {f_o} )-acrtg(\frac { g_{m1} }{ g_{m6}})-arctg(\frac{g_{m1}} {C_c}\frac{C_L+C_1}{g_{m6}})=60 
\end{equation}

From this equation we get $g_{m6} > 2mS $ which is quite high , but using this stage\footnote{Which gives a positive zero} we have deal with some trade-offs.

From the transconductance we can find the current of the second stage and the form factor of M6 and m7.


\begin{center}
\begin{circuitikz} 
    
    \ctikzset{tripoles/mos style/arrows}
    \draw (-2,-1) node[ pmos ] {}; %input sx
   \draw (2,-1) node[ pmos,xscale = -1 ] {};  %input dx
   \draw (0,0.75) node[ pmos ] {}; %tail
   \draw (-2,-2) node[ nmos,xscale = -1 ] {}; %mirror sx
   \draw (2,-2) node[ nmos ] {};  %mirror dx
   \draw (-2,0) to[short] (2,0);
   \draw (-2,0) to[short] (-2,-0.5);
   \draw(2,0) to[short] (2,-0.5);
   \draw(-2,-1.5) to[short] (0,-1.5);

   \draw(-2.5,1.5) to[short] (2.5,1.5);
   
   \draw(-1.1,-2) to[short] (1.1,-2);
   \draw(-2.5,-2.77) to[short] (2.5,-2.77);
   \draw(-2.5,-2.77) to[short] (2.5,-2.77);
   \draw(0,-1.5) to[short] (0,-2);
   \draw (0,-2.77) -- (0,-2.77) node[ground]{};

\end{circuitikz} \end{center}




% chapter part_1 (end)