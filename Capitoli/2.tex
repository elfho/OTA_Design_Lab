\chapter{Part 2} % (fold)
\label{cha:part_2}

Since we were asked to reduce the power consumption, we needed to use a different compensation technique.

We decided to use the Ahuja compensation technique embeeding the current buffer in the first stage , in order to use the already existing current to bias the current buffer.

\section{Gain, Bandwith and Stability}

\subsection{Gain} % (fold)
\label{sec:gain}

For the gain , the current buffer will act as cascode for the active load of the input pair improving the gain.

For this design we decided to split the gain in the following way:

\begin{equation}
	G_{1^{st}_Stage}=286
\end{equation}

\begin{equation}
	G_{2^{nd}_{Stage}}=35
\end{equation}


For the first step we neglect the contribution of the cascoded load considering only $R_{01}$ as load.

We obtain $L_1=1.5 \mu m$ , but it turns out that with this sizing , the contribution of the cascoded load in parallel to $R_{01}$ lead us out of the specification.

At the end we decided to set $L_{1,2} = 1.6 \mu m$ which gives us a gain of 300.

For the moment we set $L_3=L_4$ to the minimum length.

We set the Overdrive voltages as :

\begin{itemize}
	\item $V_{ov_1}=V_{ov_2}=0.2$ to have the maximum transconductance, in order to maximize the gain and reduce the noise of the mirror.

	\item $V_{ov_3}=V_{ov_4}=0.4$ in order to minimize the input referred noise contribution of the mirror couple.

	\item $V_{ov_5}=0.2$ we made this choice in order to use the same reference used by the second stage , this will let us to have only one reference generation circuit.

	\item $V_{ov_cas1}=V_{ov_cas2}=0.2$ , the lower possible in order to not degrade the common mode input voltage range.

	\item $V_{ov_6}=V_{ov_7}=0.2$ to have our amplifier almost rail-to-rail.

\end{itemize}

\subsection{Bandwith} % (fold)
\label{sub:bandwith}

As in the previous design , assuming for the moment $C_c=1pF$ we get $g_{m1}= 250 \mu S$

% subsection bandwith (end)



\subsection{Phase Margin} % (fold)
\label{sub:phase_margin}

The compensation technique will not change the frequency of the first pole, and will give a negative zero at $f_z= 	\frac{g_{mCAS}}{2 \pi C_c}$ which improves our phase margin.

\begin{equation}
	\phi _m= 180 - 90 + arctg(\frac{g_{m1}}{g_{mCAS}}) - arct(DIOCANE)
\end{equation}


% subsection phase_margin (end)_sub
% subsubsection _sub (end)
% section gain (end)
% chapter part_2 (end)